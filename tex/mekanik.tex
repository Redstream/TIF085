\section{Mekanik}

\subsection{Rörelse}
    \begin{math}
        v = v_0 + at
    \end{math} \\[2pt]
    \begin{math}
        s = v_0 t + \frac{1}{2} at^2
    \end{math} \\[2pt]
    \begin{math}
        v^2 = v_0^2 + 2as
    \end{math} \\[2pt]
    \begin{math}
        s = \frac{(v + v_0)t}{2}
    \end{math} \\[2pt]

\subsection{Newtons lagar}
\begin{enumerate}
    \item
        Tröghetslagen. En kropp förblir i vila eller konstant hastighet så länge som summan av alla yttre krafter är noll.
    \item
        \begin{math}
            F = \frac{dp}{dt}, \tab p = mv, \tab F = m \cdot a
        \end{math} \\[2pt]
    \item
        Krafter uppträder i par. Om föremål A utsätter föremål B för en viss kraft kommer B utsätta A för samma kraft men åt motsatt håll.
\end{enumerate}

\subsection{Energi}

\subsection{Arbete}

\subsection{Rörelsemängd}

\subsection{Cirkulärrörelse}
