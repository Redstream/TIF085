\section{Mekanik}

\subsection{Friläggning}
Friktion:

\subsection{Kraft}
Friktion: 
\begin{math}
    F_f = N \cdot \mu_k
\end{math} \\[2pt]
Gravitationskraften: 
\begin{math}
    F_g = mg
\end{math} \\[2pt]

\subsection{Rörelse}
    \begin{math}
        v = v_0 + at
    \end{math} \\[2pt]
    \begin{math}
        s = v_0 t + \frac{1}{2} at^2
    \end{math} \\[2pt]
    \begin{math}
        v^2 = v_0^2 + 2as
    \end{math} \\[2pt]
    \begin{math}
        s = \frac{(v + v_0)t}{2}
    \end{math} \\[2pt]
    Om accelerationen är konstant kan man få fram formler för hastighet och sträcka. \\
    \begin{math}
        a(t) = \frac{dv}{dt} = a
    \end{math} \\[2pt]
    \begin{math}
        v(t) = \int a(t) \ dt = \frac{ds}{dt} = a \cdot t + C
    \end{math} \\[2pt]
    \begin{math}
        s(t) = \int v(t) \ dt = \frac{a \cdot t^2}{2} + C + D
    \end{math} \\[2pt]
    där C är starthastigheten och D är startsträckan.

\subsection{Newtons lagar}
\begin{enumerate}
    \item
        Tröghetslagen. En kropp förblir i vila eller konstant hastighet så länge som summan av alla yttre krafter är noll.
    \item
        \begin{math}
            F = \frac{dp}{dt}, \tab p = mv, \tab F = m \cdot a
        \end{math} \\[2pt]
    \item
        Krafter uppträder i par. Om föremål A utsätter föremål B för en viss kraft kommer B utsätta A för samma kraft men åt motsatt håll.
\end{enumerate}

\subsection{Energi}
Potentiell energi:
\begin{math}
    E_p = mgh
\end{math} \\[2pt]
Rörelseenergi:
\begin{math}
    E_v = \frac{mv^2}{2}
\end{math} \\[2pt]
Om det inte finns någon förlust av energi (t.ex friktion) kan man utnyttja bevarandet av den mekaniska energin, dvs 
\begin{math}
    E_v = E_p
\end{math} \\[2pt]

\subsection{Arbete}
Friktion: 
\begin{math}
    W_{fr} = F_f \cdot s
\end{math} \\[2pt]

Lyfta någonting är ett arbete, använd formeln 
\begin{math}
    W = mgh
\end{math} \\[2pt]

Att öka ett föremåls hastighet är att utföra ett arbete. För att räkna ut det är det 
\begin{math}
    W = \frac{mv^2}{2}
\end{math} \\[2pt]

\subsection{Rörelsemängd}
\begin{math}
    m_1 \cdot v_1 + m_2 \cdot v_2 = (m_1 + m_2) \cdot v_3
\end{math} \\[2pt]

\subsection{Cirkulärrörelse}
\begin{math}
    a = \frac{m \cdot v^2}{r}
\end{math} \\[2pt]
